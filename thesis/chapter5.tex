\chapter{Conclusion}
\label{ch:5}

A search for rare charm decays has been presented in this thesis. These rare charm decays are an example of FCNC, which serve as excellent probes into new BSM physics models due to their suppressed SM nature. Previous work in FCNC probes often omits charm decays and focuses on strange or bottom decays. Specifically, this thesis calculated a limit on the branching fraction of the $D^0 \to \mu^+ \mu^-$ decay, denoted $\mathcal{B}(D^0 \to \mu^+ \mu^-)$. 

The analysis in this thesis had two major challenges. The first is the large amount of combinatorial background compared to the small amount of signal. The second is the presence of muon fakes, causing pions to be misreconstructed as muons and faking $D^0 \to \mu^+ \mu^-$ events. To address these obstacles, this thesis focused on $D^0$ mesons that decayed from $D^*$ mesons. This allowed for a more robust reconstruction of the signal events which kept combinatorial backgrounds low. Additionally, this thesis leveraged the well known quantity $\mathcal{B}(D^0 \to \pi^+ \pi^-)$ to use the $D^0 \to \pi^+ \pi^-$ decay as a normalization channel. This allowed for the ability to calculate $\mathcal{B}(D^0 \to \mu^+ \mu^-)$ without actually knowing how many $D^0$ mesons were produced. Additionally, this normalization channel caused many MC mismodeling effects that would need corrections to cancel, allowing for a simpler analysis. The primary data used in this analysis consisted of two proton-proton collision data samples taken during 2022 and 2023 using the CMS detector at the LHC. One sample was constructed using a ZeroBias trigger while the other was collected using a Dimuon trigger with a center of mass energy at $\sqrt{s} = 13.6$ TeV and integrated luminosity of $51.415\; \text{nb}^{-1}$ and $64.525\; \text{fb}^{-1}$ respectively. 

No obvious exceed is observed. The final upper limit on the branching fraction at a $90(95)\%$ confidence level was found to be 
\begin{equation}
    \mathcal{B}(D^0 \to \mu^+ \mu^-) < 2.1(2.4) \times 10^{-9} 
\end{equation}
with the final fit result displayed in figure \ref{fig:final_observed_fit}. 

This upper limit outperformed the previous world-best limit calculated by LHCb at a $90\%$ confidence level to be $3.1 \times 10^{-9}$, allowing for new phase-space limitations on BSM physics. 
% TODO: add reference above and in introduction

In addition, this analysis made many other key contributions in the process of deriving its final result. Namely, this analysis performed one of the most exhaustive fake rate studies to date, presenting a result and continuing development on a method which will be able to be used by many other analysis which involve fake rates. Additionally, this analysis was the first to use the new inclusive dimuon trigger developed at CMS. It demonstrated the benefits of the new trigger, illuminating the opportunities for its use in low-mass muon pair data. 

