\chapter{Conclusion}
\label{ch:5}

\vspace{-5mm}

A search for rare charm decays has been presented in this thesis. These rare charm decays are examples of FCNCs, which serve as excellent probes into BSM physics models due to their suppressed SM nature. Previous FCNC studies often omit charm decays and focus on strange or bottom decays. Specifically, this thesis calculated a limit on the branching fraction of the $D^0 \to \mu \mu$ decay, denoted $\mathcal{B}(D^0 \to \mu \mu)$. 

The analysis in this thesis had two major challenges. The first was the large amount of combinatorial background compared to the small amount of signal. The second was the presence of muon fakes, causing pions to be misreconstructed as muons and faking $D^0 \to \mu \mu$ events. To address these obstacles, this thesis focused on $D^0$ mesons that decayed from $D^*$ mesons. This allowed for a more robust reconstruction of the signal events which kept combinatorial backgrounds low. Additionally, this thesis leveraged the well-known quantity $\mathcal{B}(D^0 \to \pi^\pm \pi^\pm)$ to use the $D^0 \to \pi^\pm \pi^\pm$ decay as a normalization channel. This allowed for the ability to calculate $\mathcal{B}(D^0 \to \mu \mu)$ without actually knowing how many $D^0$ mesons were produced. Furthermore, this normalization channel caused a reduction in many systematic uncertainties, allowing for a more precise measurement. The primary data used in this analysis consisted of proton-proton collision data samples taken during 2022 and 2023 using the CMS detector at the LHC. One sample was constructed using a prescaled zero bias trigger while the other was collected using a dimuon trigger with a center of mass energy at $\sqrt{s} = 13.6$ TeV and corresponding to an integrated luminosity of $51.4\; \text{nb}^{-1}$ and $64.5\; \text{fb}^{-1}$ respectively. 

No obvious excess is observed. The final upper limit on the branching fraction at a $90(95)\%$ confidence level was found to be 
\begin{equation}
    \mathcal{B}(D^0 \to \mu \mu) < 2.1(2.4) \times 10^{-9} 
\end{equation} 

This upper limit outperformed the previous world-best limit calculated by LHCb at a $90\%$ confidence level to be $3.1 \times 10^{-9}$ \cite{ref:lhcb_2023}, placing new constraints on BSM physics.
% TODO: add reference above and in introduction

In addition, this analysis made other key contributions in the process of deriving its final result. Namely, this analysis performed one of the most exhaustive fake rate studies to date, presenting a result and continuing development on a method which will be able to be used by many other analyses which involve muon fake rates. This analysis was also the first to use the new inclusive dimuon trigger developed at CMS. It demonstrated the benefits of the new trigger, illuminating the opportunities for its use in low-mass dimuon data. 