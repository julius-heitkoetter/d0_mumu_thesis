\chapter{Introduction}

TODO: add hook to the beginning of this introduction

One promising approach to searching for new physics is the study of rare processes that are highly suppressed in the Standard Model (SM). Such processes offer a unique opportunity to identify small deviations from SM predictions that may indicate contributions from beyond the Standard Model (BSM) physics. One such suppressed process is Flavor Changing Neutral Currents (FCNC), which are hadronic decay processes that are not allowed to leading order in the SM as a consequence of none of the two neutrally charged electroweak bosons ($Z$ and $\gamma$) being flavor changing. However, there are many beyond the standard model (BSM) theories that allow for FCNCs, such as Super Symmetry (SUSY) or Flavor Non-Universal Z' Bosons.

Of the 6 quarks in the SM ($u,d,c,s,t,b$), the up/down quarks cannot decay because they are the lightest generation quarks. The top quark due to its large mass is extremely short lived and decays into a $W$ boson and a bottom quark before it can form hadrons. Decays of hadrons composed of bottom and strange quarks have been extensively researched, while decays of charmed hadrons receive less attention due to a smaller branching fraction in the $c \to u$ decay relative to the $b\to s$ or $s \to d$ decays. This makes it particularly interesting to study FCNCs of charmed, or $D$, mesons.

One such accelerator experiment is the Compact Muon Solenoid (CMS), one of two large general-purpose detectors at the Large Hadron Collider (LHC). CMS specializes in muon detection, making a dimuon ($\mu^+\mu^-$) final state a useful choice for a rare decay. Therefore, this thesis looks to search for the rare decay of a neutrally changed $D$ meson into a dimuon final state, or more specifically set an upper limit on $\mathcal{B}(D^0 \to \mu^+ \mu^-)$.

Previous work studying $\mathcal{B}(D^0 \to \mu^+ \mu^-)$ has been most successfully done by the Large Hadron Collider beauty (LHCb) experiment, achieving an upper limit of the branching fraction at $3.1\times 10^{-9}$ at a $90\%$ confidence level. Current leading theoretical work places the SM prediction of the branching fraction at $3\times 10^{-13}$, leaving an unexplored region of 4 orders of magnitude to probe for new physics.

During this thesis project, I will run an analysis to measure the upper limit on $\mathcal{B}(D^0 \to \mu^+ \mu^-)$. The main challenge of the analysis is the large amount of background events, making it difficult to detect signal events. The main approach to reduce the background is to to look at the cascade decays of $(D^*)^\pm \to D^0 \pi^\pm \to (\mu^+ \mu^-) \pi^\pm$, which produces a characteristic additional pion that helps to tag and reconstruct the event more cleanly. 

This technique, while increasing the signal-to-background ratio significantly has two main obstacles. The first is the muon fake rate, or the decay of $\pi \to \mu \gamma$, where $\gamma$ has low enough energy that the muon and pion have essentially the same 4-momentum, causing the detector to falsely reconstruct a muon as the product of the decay, instead of a pion. This causes $D^0 \to \pi^+ \pi^-$ decays (a common decay) to be misreconstructed as $D^0 \to \mu^+ \mu^-$ events. The second obstacle is that canonically, to get $\mathcal{B}(D^0 \to \mu^+ \mu^-)$, one would count the number of $D^0 \to \mu^+ \mu^-$ events, $N_{D^0 \to \mu^+ \mu^-}$, and divide by the number of $D^0$ mesons produced in the detector, $N_{D^0}$. However, $N_{D^0}$ is not well known at CMS. Instead, $\mathcal{B}(D^0 \to \pi^+ \pi^-)$ is very well known, meaning that if one measures $N_{D^0 \to \pi^+ \pi^-}$ in a normalization channel, then one can construct
$$
\mathcal{B}(D^0 \to \mu^+ \mu^-) = \mathcal{B}(D^0 \to \pi^+ \pi^-) \frac{N_{D^0 \to \mu^+ \mu^-}}{N_{D^0 \to \pi^+ \pi^-}} \times \text{efficiency corrections,}
$$
where the dominating challenge lies in properly tracking the efficiency corrections. 

The thesis is structured as follows: Chapter 1 discusses the theoretical motivation for rare charm decays, Chapter 2 describes the LHC and CMS detector, and Chapter 3 presents the analysis. The analysis consists of three key components: (1) measuring the muon fake rate, (2) implementing a multivariate classification to distinguish background, signal, and normalization events, and (3) performing an unbinned maximum likelihood fit to extract the signal while accounting for efficiency corrections.

TODO: change language from "I will" to "I did" and give longer breakdown at the end on the structure of the thesis