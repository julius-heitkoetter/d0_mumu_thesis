\chapter{Introduction}

The Standard Model (SM) of particle physics is one of the most successful scientific theories ever developed by humans. It accounts for the electromagnetic, weak, and strong forces, intertwining them in a single mathematical framework. It predicts the behavior of fundamental physics with incredible precision, as demonstrated in numerous predictions such as the discovery of the Higgs boson and the measurement of the magnetic moment of the electron. Furthermore, these predictions hold over a large range of energies and length scales, earmarking the success of the theory.

However, we know that the SM cannot be the full story. For example, it contains no explanation for dark matter, which has been found to make up the majority of matter in the universe. It also does not explain the large matter-antimatter symmetry that allows for the stable matter we rely on to exist. Lastly, and perhaps most alarming, it is inconsistent when combined with gravity in a quantum gravity model. 

Even within the scope of the SM, it is unclear why there are 3 generations of quarks, why the particles each have the masses that they do, or why only certain interactions are allowed. Many of these questions can be explained away by considering the parameters they require (such as the quark masses) as fundamental constants of the universe. However, many of these puzzles could be clues into a deeper understanding of the structure of the SM. In the modern understanding of physics, this \textit{structure} refers not to the equations of motion, but rather to the exact geometrical symmetries of our universe. For example, the SM can be derived from a $SU(3) \times SU(2) \times U(1)$ symmetry structure, as outlined later in this thesis. Finding deeper structures is therefore often equivalent to probing for new symmetries, such as is done with a common extension of the SM known as the Super Symmetry (SUSY) model. 

Finding deeper structure does not always begin with large signature discoveries. Often, it begins with considering edges of phase space where the SM heavily suppresses processes, leading to not only robustly small but also incredibly fragile theoretical expectations. Even extremely small signals measured in these places can lead to incredible insight into the structure of our universe.

Therefore, one promising approach to searching for new physics is the study of rare processes that are highly suppressed in the Standard Model (SM). Such processes offer a unique opportunity to identify small deviations from SM predictions that may indicate contributions from Beyond the Standard Model (BSM) physics. One such suppressed process is Flavor Changing Neutral Currents (FCNC), which is are hadronic decay processes not allowed at leading order in the SM as a consequence of none of the two neutrally charged electroweak bosons ($Z$ and $\gamma$) being flavor changing. However, there are many Beyond the Standard Model (BSM) theories that allow for FCNCs at tree level, such as SUSY or Flavor Non-Universal Z' Bosons.

Of the six quarks in the SM ($u,d,c,s,t,b$), the up/down quarks cannot decay because they are the lightest generation quarks. The top quark, due to its large mass, is extremely short-lived and decays into a $W$ boson and a bottom quark before it can form hadrons. Decays of hadrons composed of bottom and strange quarks have been extensively researched, while decays of charmed hadrons receive less attention due to a smaller branching fraction in the $c \to u$ decay relative to the $b\to s$ or $s \to d$ decays. This makes it particularly interesting to study FCNCs of charmed, or $D$, mesons.

One such accelerator experiment is the Compact Muon Solenoid (CMS), one of two large general-purpose detectors at the Large Hadron Collider (LHC). CMS specializes in muon detection, making a dimuon ($\mu^+\mu^-$) final state a useful choice for a rare decay. Therefore, this thesis looks to search for the rare decay of a neutrally charged $D$ meson into a dimuon final state, or more specifically set an upper limit on $\mathcal{B}(D^0 \to \mu^+ \mu^-)$.

Previous work studying $\mathcal{B}(D^0 \to \mu^+ \mu^-)$ has been most successfully done by the Large Hadron Collider beauty (LHCb) experiment, achieving an upper limit of the branching fraction at $3.1\times 10^{-9}$ at a $90\%$ confidence level \cite{ref:lhcb_2023}. Current leading theoretical work places the SM prediction of the branching fraction at $3\times 10^{-13}$ \cite{ref:burdman_2002}, leaving an unexplored region of 4 orders of magnitude to probe for new physics.

In this thesis, I run an analysis to measure the upper limit on $\mathcal{B}(D^0 \to \mu^+ \mu^-)$. The main challenge of the analysis is the large amount of background events, making it difficult to isolate signal events. The main approach to reduce the background is to look at the cascade decays of $(D^*)^\pm \to D^0 \pi^\pm \to (\mu^+ \mu^-) \pi^\pm$, which produces a characteristic additional pion that helps to tag and reconstruct the event more cleanly. 

This technique, while increasing the signal-to-background ratio, has two main obstacles. The first is the muon fake rate, or the decay of $\pi \to \mu \nu$, where $\nu$ has low enough energy that the muon and pion have essentially the same 4-momentum, causing the detector to falsely reconstruct a muon as the product of the decay, instead of recognizing a pion. This causes the more common $D^0 \to \pi^+ \pi^-$ decay to be misreconstructed as a $D^0 \to \mu^+ \mu^-$ event. The second obstacle is that canonically, to get $\mathcal{B}(D^0 \to \mu^+ \mu^-)$, one would count the number of $D^0 \to \mu^+ \mu^-$ events, $N_{D^0 \to \mu^+ \mu^-}$, and divide by the number of $D^0$ mesons produced in the detector, $N_{D^0}$. However, $N_{D^0}$ is not well known at CMS. Instead, $\mathcal{B}(D^0 \to \pi^+ \pi^-)$ is very well known, meaning that if one measures $N_{D^0 \to \pi^+ \pi^-}$ in a normalization channel, then one can construct
$$
\mathcal{B}(D^0 \to \mu^+ \mu^-) = \mathcal{B}(D^0 \to \pi^+ \pi^-) \frac{N_{D^0 \to \mu^+ \mu^-}}{N_{D^0 \to \pi^+ \pi^-}} \times \text{efficiency correction ratio}
$$
where the dominating challenge lies in properly tracking the efficiency correction ratio.

\bigbreak

This concludes an introduction into rare charm decays. The rest thesis is structured as follows:

Chapter \ref{ch:2} of this thesis lays the theoretical groundwork on which the remainder of the thesis is built and gives insight to why the rare decay $D^0 \to \mu^+ \mu^-$ is a valuable probe for new physics. It begins with a short survey of Quantum Field Theory (QFT), outlining how the adoption of special relativity into quantum mechanics gives rise to a QFT characterized by a Lagrangian and resulting correlation functions that can be calculated perturbatively, using tools such as path integrals, Feynman diagrams, and the Lehmann Symanzik Zimmermann (LSZ) reduction formulation to extract scattering amplitudes. The chapter then describes the SM, beginning with its local gauge symmetry structure and building the fundamental particles and interactions from the resulting Lagrangian, with special attention paid to the electroweak sector and the mechanism by which the Higgs field gives mass to the $W^\pm$ and $Z$ bosons. This chapter also applies the theoretical groundwork to motivate the $D^0 \to \mu^+ \mu^-$  search by showing that FCNCs are forbidden at tree level in the SM, with loop contributions predicting a branching fraction of $\simeq 10^{-13}$. It uses this discussion to conclude in outlining how the $D^0 \to \mu^+ \mu^-$ decay can be used as a powerful probe of new physics. 

Chapter \ref{ch:3} of this thesis describes the experimental foundation used to measure the $D^0 \to \mu^+ \mu^-$ branching fraction, beginning with a summary of the Large Hadron Collider at CERN. The chapter then describes the CMS detector, starting with an overview of its geometry and coordinate system before detailing each of its subsystems. For each of these subsystems, the chapter outlines how the engineering of the detectors impacts the physics read-out from them. The next section explains how the CMS experiment uses a trigger system to balance the large volume of data with the need to capture rare particle interactions. Finally, the chapter discusses how raw data is reconstructed into particle events and how simulation is used for analysis and to benchmark detector response and acceptance.

Chapter \ref{ch:4} of this thesis describes the experiment performed to measure the branching fraction of the $D^0 \to \mu^+ \mu^-$ decay. It begins with an overview of the analysis strategy, before diving into the datasets and simulation samples. It then describes the reconstruction and event selection methods used to generate the signal and normalization datasets, as well as details the multivariate analysis boosted decision tree framework used to better identify signal events. Then, the chapter covers the calculation of $N_{D^0 \to \mu^+ \mu^-}$ and $N_{D^0 \to \pi^+ \pi^-}$ events using an unbinned maximum likelihood fit. Then, this chapter covers the efficiency corrections derived from simulation samples, namely the trigger efficiency, the muon reconstruction efficiency, and the muon fake rate. Lastly, the chapter systematic uncertainties as well as the $CL_s$ method used to calculate the final result, before presenting the final result. 

Chapter \ref{ch:5} of this thesis gives a short conclusion on the findings of the analysis performed in this thesis.

This thesis ends with a conclusion of the experimental study, summarizing and giving commentary on its results. 

% TODO: cite the LHCb and scan for any other citaiton