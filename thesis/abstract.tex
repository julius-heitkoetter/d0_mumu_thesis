% From mitthesis package
% Version: 1.01, 2023/06/19
% Documentation: https://ctan.org/pkg/mitthesis
%
% The abstract environment creates all the required headers and footnote. 
% You only need to add the text of the abstract itself.
%
% Approximately 500 words or less; try not to use formulas or special characters
% If you don't want an initial indentation, do \noindent at the start of the abstract

A search for the rare charm decay $D^0 \to \mu^+ \mu^-$ is presented. This decay is an example of a flavor changing neutral current, which serves as an excellent probe into new Beyond the Standard Model physics due to its suppressed Standard Model nature. The analysis is performed using proton-proton collision data collected from the CMS detector at the Large Hadron Collider in 2022 and 2023, with two datasets at a center of mass energy of $\sqrt{s} = 13.6$ TeV and integrated luminosity of $51.415\; \text{nb}^{-1}$ and $64.525\; \text{fb}^{-1}$. The analysis uses $D^0$ mesons originating from $D^{*\pm} \to D^0 \pi^\pm$ decays, which allow for better signal selection and less contribution from combinatorial backgrounds. The analysis uses a normalization approach with respect to the $D^0 \to \pi^+ \pi^-$ decay to calculate its results, which allows for strategic cancelations of systematic uncertainties and lack of need for a $D^0$ cross-section. Pions decaying in-flight to muons cause a significant difficulty for the analysis which much be carefully studied and understood. No obvious exceed is observed. Upper limits on the $\mathcal{B}(D^0 \to \mu^+ \mu^-)$ decay at a $90(95)\%$ confidence level are found to be $2.1(2.4)\times10^{-6}$, outperforming the current world-best limit by $35\%$. 